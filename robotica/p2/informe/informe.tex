\documentclass{article}
\usepackage[utf8]{inputenc}
\usepackage[margin=1in,left=1.5in,includefoot]{geometry}
\usepackage{booktabs}
\usepackage{graphicx}

% Header & Footer Stuff

\usepackage{fancyhdr}
\pagestyle{fancy}
\lhead{Fundamentos de Robótica Intelixente}
\rhead{614G030302425}
% \fancyfoot{}
% \lfoot{Pablo Chantada Saborido \& José Romero Conde}
% \fancyfoot[R]{}

% The Main Document
\begin{document}
\begin{center}
    \LARGE\bfseries PRÁCTICA II\\
    \small Pablo Chantada Saborido \& José Romero Conde
    \line(1,0){430}
\end{center}


\section*{Resolución de los ejercicios}
En ambos ejercicios, el control del robot ocurre en la llamada:

\par{\begin{center}\texttt{robobo.whenANewColorBlobIsDetected(blobDetectedCallback)}\end{center}}

\noindent y por tanto en la correspondiente función \texttt{blobDetectedCallback}, esta es igual para ambos casos. Este callback a su vez llama a otras dos funciones para manejar el comportamiento del robot: \texttt{centerToAColor(blob)} y \texttt{moveToAColor(blob)}. Las cuales están diferentemente implementadas para cada ejercicio.

\subsection*{Ejercicio 1 - \emph{Control Proporcional}}
\subsubsection*{Función \texttt{centerToAColor(blob)}}
Dentro de esta función, el robot centra el blob según el error de los sensores con el centro \textit{(se considera centro como el valor medio del rango [0,100])}. Añadimos además un leve margen de error para evitar bucles infinitos por precisión \footnote{El robot se queda moviendose de derecha a izquierda por no encontrar el valor exacto.}. La velocidad a la que gira para centrarse es fija, independiente del error. 

\subsubsection*{Función \texttt{moveToAColor(blob)}}
Se implementa el control proporcional. Por tanto, tenemos un coeficiente \texttt{KPav}
y una variable \texttt{P} que es igual al sensor IR frontal. Con esto, calculamos la corrección $\varepsilon_t \; (:= \; \texttt{P} \times \texttt{KPav} )$  de forma que \[ v_t := v_{t-1} - \varepsilon_t \] siendo $v_t$ la velocidad del robot en el instante $t$.

\subsection*{Ejercicio 2 - \emph{Control PID}}
En el anterior caso cuando \texttt{P} \textit{(la variable que carga el error actual)} era menor que un cierto valor, la ejecución se paraba; ahora se llama a otra subrutina \texttt{blob\_is\_close(speed, distance)} que se encarga de mover el blob si estamos seguros de estar cerca; por último, se detiene la ejecución. 
\subsubsection*{Función \texttt{centerToAColor(blob)}}
Ahora, la velocidad de giro para centrarse sí depende del error, siguiendo un esquema PID. Para ello, definimos $P_i$ como el error en el eje $x$ (\texttt{abs(blob.posx - 50})). Adicionalemente ahora cargamos con dos variables más (y sus coeficientes), que se definen de la siguiente forma \[I_t = \sum_{i \in [0,t]} P_i \] \[D_t = P_t - P_{t-1}\] con esto, definimos la nueva corrección, que usamos para el movimiento del robot para que el blob esté constantemente centrado: \[\varepsilon_t := P_t\times K_p + I_t \times K_I + D_t \times K_D\]
\subsubsection*{Función \texttt{moveToAColor(blob)}}
Seguimos la misma estrategia de PID descrita, pero ahora siendo \texttt{P} como se definió en el ejercicio 1. 
\section*{Ajuste de controladores}
Los valores de los controladores han sido ajustados manualmente, viendo el comportamiento en el simulador. Únicamente se han usado los mapas indicados en la guía; en otros escenarios podría ser necesario un ajuste a estos.
\end{document}

