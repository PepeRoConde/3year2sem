\documentclass{article}
\usepackage[utf8]{inputenc}
\usepackage[margin=1in,left=1.5in,includefoot]{geometry}
\usepackage{booktabs}
\usepackage{graphicx}
\usepackage{listings}
\usepackage{xcolor}
\usepackage{hyperref}


\definecolor{codegreen}{rgb}{0,0.6,0}
\definecolor{codegray}{rgb}{0.5,0.5,0.5}
\definecolor{codepurple}{rgb}{0.58,0,0.82}
\definecolor{backcolour}{rgb}{0.95,0.95,0.92}

\lstdefinestyle{mystyle}{
    backgroundcolor=\color{backcolour},   
    commentstyle=\color{codegreen},
    keywordstyle=\color{blue},
    numberstyle=\tiny\color{codegray},
    stringstyle=\color{codepurple},
    basicstyle=\ttfamily\small,
    breakatwhitespace=false,         
    breaklines=true,                 
    captionpos=b,                    
    keepspaces=true,                 
    numbers=left,                    
    numbersep=5pt,                  
    showspaces=false,                
    showstringspaces=false,
    showtabs=false,                  
    tabsize=2
}

\lstset{style=mystyle}
% Header & Footer Stuff
\usepackage{fancyhdr}
\pagestyle{fancy}
\lhead{Fundamentos de Robótica Inteligente}
\rhead{614G030302425}
% \fancyfoot{}
% \lfoot{Pablo Chantada Saborido \& José Romero Conde}
% \fancyfoot[R]{}

% The Main Document
\begin{document}
\begin{center}
    \LARGE\bfseries PRÁCTICA III\\
    \small Pablo Chantada Saborido \& José Romero Conde
    \line(1,0){430}
\end{center}

\section*{Introducción}

Como configuración de la aplicación de Robobo utilizamos "Random Behaviour in Worlds" con "Horizontal Flip Front Camera" desactivado. Para evitar la repetición de código, implementamos una clase base \texttt{RoboboController} que implementa la funcionalidad común, y clases derivadas para cada tipo de interfaz: \texttt{TextRoboboController}, \texttt{VoiceRoboboController}, \texttt{ChatGPTRoboboController} y \texttt{VectorRoboboController}.

Cada controlador implementa su propio método \texttt{get\_command()} para obtener instrucciones del usuario, mientras que comparten la lógica de movimiento, emociones y efectos.

\section*{Ejercicio 1: Movimiento con Teclado}

Este ejercicio implementa un controlador básico del robot Robobo mediante teclado utilizando la biblioteca \texttt{pynput}.

Control: W (adelante), S (atrás), A (izquierda), D (derecha), Q (detener y salir).

El robot utiliza una velocidad constante (SPEED=10) y detiene los motores al soltar cualquier tecla. Los movimientos de izquierda y derecha giran el robot de forma estacionaria.

\section*{Ejercicio 2: Movimiento con Cuadro de Texto}

Este ejercicio permite controlar el robot Robobo mediante comandos de texto introducidos en una interfaz gráfica con cuadros de diálogo.

Movimientos básicos: forward, back, left, right.

Movimientos compuestos: forward-left, forward-right, back-left, back-right.

Admite parámetros adicionales:
\begin{itemize}
\item speed X: Establece velocidad a X
\item time Y: Establece duración a Y segundos
\end{itemize}

La implementación se basa en la clase \texttt{TextRoboboController} que hereda de la clase base \texttt{RoboboController}, modificando únicamente la obtención de los comandos mediante el diálogo anteriormente mencionado.

\section*{Ejercicio 2b: Movimiento con STT}

Apartado opcional del ejercicio 2 que utiliza reconocimiento de voz en español y añade control de emociones y LEDs.

Movimientos: delante, atrás, izquierda, derecha.

Emociones: feliz, triste, enfadado.

Colores LED: rojo, verde, azul.

Utilizando la biblioteca \texttt{speech\_recognition} con reconocimiento Google (\texttt{language="es"}). Cambiamos el idioma de los comandos para facilitar la comprobación de su funcionalidad, el tiempo de recepción de comandos debe ser cambiado en el propio código; inicialmente está puesto a 4,5 segundos.

\section*{Ejercicio 3: Definición de prompt para ChatGPT}

Este ejercicio consiste en la definición de un prompt específico para ChatGPT que permite extraer comandos de movimiento para el robot a partir de frases en lenguaje natural. A continuación se muestra el contenido del prompt que utilizamos para el entrenamiento:

\begin{lstlisting}[language=markdown]
Necesito que extraigas comandos de movimiento para un robot a partir de frases en lenguaje natural. 

Los comandos permitidos son:
- forward: para mover hacia adelante
- back: para mover hacia atras
- left: para girar a la izquierda
- right: para girar a la derecha
- stop: para detener el movimiento

Tambien puedes combinar los comandos de direccion para movimientos compuestos como:
- forward-left: para avanzar girando a la izquierda
- forward-right: para avanzar girando a la derecha
- back-left: para retroceder girando a la izquierda
- back-right: para retroceder girando a la derecha

Ademas, puedes incluir parametros:
- speed X: donde X es un numero entero que indica la velocidad
- time Y: donde Y es un numero que indica los segundos de movimiento

Ejemplos:
- "Avanza un poco": forward
- "Retrocede a velocidad 30": back speed 30
- "Gira a la derecha durante 3 segundos": right time 3
- "Avanza girando a la izquierda a velocidad 25": forward-left speed 25

Responde UNICAMENTE con el comando extraido, sin explicaciones ni formato adicional.
\end{lstlisting}

La ejecución del entrenamiento se puede comprobar al \href{ej3}{final del informe}.

\section*{Ejercicio 4: Movimiento con ChatGPT}

Este ejercicio integra ChatGPT para interpretar comandos en lenguaje natural y convertirlos en instrucciones para el robot.

El usuario introduce una frase (ej: "Avanza un poco"), ChatGPT la procesa y devuelve un comando (indicado en el ejercicio 3, ej: "forward"), que el controlador ejecuta. Utilizamos temperatura 0 en este caso, ya que queremos únicamente los comandos establecidos.

\newpage

\section*{Ejercicio 5: Movimiento con vector de ChatGPT}

Este ejercicio, a diferencia del anterior, genera un vector de control completo para el robot.

Vector de control: [vx, vy, emoción, sonido, texto]

Donde vx, vy son las velocidades de las ruedas (-100 a 100), emoción y sonido pueden ser "happy", "sad" o "angry", y texto es un mensaje que el robot emite.

El controlador ejecuta todos los aspectos simultáneamente y detiene el robot después de 3 segundos. Fue necesario adaptar el entrenamiento del ejercicio 3 usando el siguiente texto:

\begin{lstlisting}
Necesito que generes comandos de control para un robot Robobo a partir de frases en lenguaje natural.

Tu respuesta DEBE ser un vector de 5 elementos con el siguiente formato exacto: [vx, vy, emocion, sonido, texto]

Donde:
- vx y vy son valores numericos entre -100 y 100 que representan las velocidades de las ruedas derecha e izquierda, usa 20 como valor por defecto
- emocion es un string que debe ser uno de los siguientes: "happy", "sad", "angry"
- sonido es un string que debe ser uno de los siguientes: "happy", "sad", "angry"
- texto es un string con el mensaje que el robot debe decir

Ejemplos de comandos y sus respuestas:
- "Avanza rapido con cara feliz": [50, 50, "happy", "happy", "Avanzando a toda velocidad!"]
- "Gira a la derecha lentamente con cara triste": [-20, 20, "sad", "sad", "Girando a la derecha..."]

Para los movimientos, considera:
- Avanzar: valores positivos iguales en vx y vy
- Retroceder: valores negativos iguales en vx y vy
- Girar a la derecha: vx positivo, vy negativo
- Girar a la izquierda: vx negativo, vy positivo                        
- Velocidad rapida: valores cercanos a 100
- Velocidad lenta: valores cercanos a 20

Responde UNICAMENTE con el vector, sin explicaciones ni texto adicional.
\end{lstlisting}


\newpage
\begin{figure}
    \centering
    \includegraphics[width=1\linewidth]{chat_gpt_response.png}
    \caption{Entrenamiento de ChatGPT para movimiento de Robobo}
    \label{fig:ej3}
\end{figure}

\end{document}