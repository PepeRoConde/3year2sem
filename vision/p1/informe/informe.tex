\documentclass{article}
\usepackage[utf8]{inputenc}
\usepackage[margin=1in,left=1.5in,includefoot]{geometry}
\usepackage{booktabs}
\usepackage{graphicx}
\usepackage{hyperref}

% Header & Footer Stuff

\usepackage{fancyhdr}
\pagestyle{fancy}
\lhead{Visión por Computador Aplicada}
\rhead{614G030332425}
% \fancyfoot{}
% \lfoot{Pablo Chantada Saborido \& José Romero Conde}
% \fancyfoot[R]{}

% The Main Document
\begin{document}
	\begin{center}
		\LARGE\bfseries PRÁCTICA II\\
		\small Pablo Chantada Saborido \& José Romero Conde
		\line(1,0){430}
	\end{center}
	
\section*{Introducción}

\section*{Clasificación Ship/No-ship}
\subsection*{Modelo base}

Describimos ahora nuestro modelo baseline, que nos valió para iterar y comparar resultados. El modelo se compone de los siguientes elementos:

\begin{itemize}
	\item  \textbf{\href{https://arxiv.org/pdf/1905.11946}{Efficienet.}} La usamos como CNN de partida. Tiene como requirimiento que las imágenes sean de tamaño $(244,244,3)$, para ello implementamos un recortador automático de \emph{el cuadrado más grande} que llamamos desde los \texttt{DataLoader} de entrenamiento y test. Aunque esta forma es claramente subóptima nos pareció adeacuada para una primera aproximación. Mas adelante se comentarán los cambios
	\item \textbf{Aumento de datos.} El aumento de datos en esta fase esencialmente consta de dos partes, por un lado, las transformaciones de \texttt{torchvision.transforms}. Nosotros usamos: volteos horizontales, fluctuaciones leves en el color, también leves trasnformaciones afines y conversión a blanco y negro con baja probabilidad. El segundo punto es que hemos recortado manualmente las imagenes de barcos de forma que queden cerntrados y sin fondo que estorbe, se han añadido estas images al conjunto de datos cuando se especificaba aumento de datos.
	\item Un \textbf{MLP} de tres capas, sobre la salida de la red. La salida de la última capa \texttt{nn.Softmax()} tiene 2 o 3 neuronas según si se quería clasificar en \{barco, no barco\} o en \{barco, barco no amarrado, barco amarrado\}. 
\end{itemize}

\subsection*{Entrenamiento y evaluación}

\section*{Clasificación Docked/Undocked}

\section*{Resultados}

\subsection*{Evaluación de modelos}

\subsection*{Análisis de resultados}

\section*{Conclusión}


\section*{Cambios para el segundo ejercicio}

























\end{document}

